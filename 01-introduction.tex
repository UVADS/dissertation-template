This is the introduction to your dissertation.  This introduction came from D. E. Brown, ``Text Mining the Contributors to Rail Accidents,'' in IEEE Transactions on Intelligent Transportation Systems, vol. 17, no. 2, pp. 346-355, Feb. 2016.  

In the 11 years from 2001 to 2012 the U.S. had more than 40,000 rail accidents with a total cost of \$45.9 M. These accidents resulted in 671 deaths and 7061 injuries. Since 1975 the Federal Railroad Administration (FRA) has collected data to understand and find ways to reduce the numbers and severity of these accidents. The FRA has set “an ultimate goal of zero tolerance for rail-related accidents, injuries, and fatalities” \cite{fra2009}.

A review of the data collected by the FRA shows a variety of accident types from derailments to truncheon bar entanglements. Most of the accidents are not serious; since, they cause little damage and no injuries. However, there are some that cause over \$1M in damages, deaths of crew and passengers, and many injuries. The problem is to understand the characteristics of these accidents that may inform both system design and policies to improve safety.

After each accident a report is completed and submitted to the FRA by the railroad companies involved. This report has a number of fields that include characteristics of the train or trains, the personnel on the trains, the environmental conditions (e.g., temperature and precipitation), operational conditions (e.g., speed at the time of accident, highest speed before the accident, number of cars, and weight), and the primary cause of the accident. Cause is a four character, coded entry based on based on 5 overall categories (discussed in Section IV). The FRA also collects data on the costs of each accident decomposed into damages to track and equipment to include the number of hazardous material cars damaged. Additionally, they report the number of injuries and deaths from each accident.

\section{Motivation}
This is a section on motivation. Text is ragged right justified, left aligned.