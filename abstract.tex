The following abstract is from the dissertation of Natalie Kupperman, PhD.  Athlete monitoring is the practice of collecting data in athletics to quantify load, both physical and mental, with the goal to reduce fatigue, optimize performance and mitigate injury risk. The influx in technology over past 15 years has given practitioners the ability to quantify these measures at an increased frequency. However, the subsequent research that followed has found limited associations between these measures and injury risk. There are many possible reasons for this paucity, one of which is the methods used to investigate these data. The purpose of this dissertation was to use three different methodological techniques to both introduce novel analysis to the field and also highlight the need for within-person design in athlete monitoring.

The purpose of manuscript 1 was to evaluate differences in external workload trends of athletes who sustained an overuse injury during sport. We did this using a case series format to evaluate differences in on-court volume, as measured by whole-body accelerometry, of three court-sport athletes and three position-matched healthy control athletes. Across sports, teams, and sexes, we found differences in accelerometer data between injured athletes and healthy matched controls in the 8 weeks leading up to the injury. The injured volleyball athlete had greater accumulated playerload per min (PL/min), jumps per minute and duration compared to the healthy athlete across the 8 weeks. For basketball, both injured athletes had greater PL/min compared to their controls, however, jumps per min were less than the healthy athletes over the 8 weeks.

The purpose of manuscript 2 was to a apply longitudinal structural equation modeling framework to athlete monitoring data with the goal of better understanding between- and within-person differences. Longitudinal athlete sport participation status and self-reported wellness were collected from 16 volleyball athletes and aggregated into weekly observations over 5 weeks. Three models – univariate latent curve model (LCM), LCM with structured residuals (SR), and LCM-SR with time varying covariates – are described. Then using simulated data based off the 16 athletes, results for each model are reported and substantively interpreted. The final model shows statistically significant effects of wellness on both the same and subsequent week athlete availability.

The purpose of manuscript 3 was to first introduce a research design using machine learning (ML) and explanatory model that is new the field of athlete monitoring and the use a case study in collegiate basketball to demonstrate the proposed research design. The research design contained 3 steps. First, data was processed using an ML algorithm designed for feature selection. Then based on the important features defined by the ML algorithm and researcher input, variables are selected, and a testable hypothesis is generated. The hypothesis was then tested using an explanatory model. The external load and readiness datasets performed best with ordinal forests, whereas the athlete self-report measures data did best with an support vector machine. Readiness had the best performance with MSE = 0.89. For the explanatory model, only the physical performance capability (PPC) model was statistically significant. This model found the odds having above average soreness decrease by 57\% for every point increase in PPC (OR=0.57, 95\% CI 0.32 to 0.97, p<0.05).

In undertaking these studies, the complexities of modeling injury were demonstrated. There is still much to be explored; however, by attempting new ways of analyzing athlete monitoring, we were able to highlight the need for continued forward thinking in methodologies, both at the level of analysis and the analytical tools employed. 